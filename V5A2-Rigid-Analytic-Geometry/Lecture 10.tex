\section{Lecture 10 -- 3rd July 2025}\label{sec: lecture 10}
Recall the following definition. 
\begin{definition}[Universally Japanese]\label{def: universally Japanese}
    Let $A$ be an integral domain. $A$ is universally Japanese if for all $\pfrak\in\spec(A)$ and all finite extensions $L/\kappa(\pfrak)$ the integral closure of $A$ in $L$ is a finitely generated $A$-module. 
\end{definition}
This is a property that holds for all affinoid algebras. 
\begin{proposition}\label{prop: affinoid implies univ Jap}
    If $A$ is an affinoid $K$-algebra then $A$ is universally Japanese. 
\end{proposition}
\begin{proof}
    Omitted. 
\end{proof}
\begin{proposition}\label{prop: maximum norm is complete}
    Let $A$ be an affinoid $K$-algebra. The norm on $A/\mathrm{Nil}(A)$ by $\Vert\cdot|A\Vert_{\max}$ is equivalent to the residual norm. In particular, it is complete. 
\end{proposition}
\begin{proof}
    Suppose that $A$ is an integral domain. By Noether normalization, we can take $A$ to be a finite $\TT_{d}$-algebra. Let $(a_{i})_{i=1}^{n}$ be a basis of $L/E$ where $L$ and $E$ are the fields of fractions of $A$ and $\TT_{d}$, respectively. Since $A$ is a finitely generated $\TT_{d}$-module, there is $f\in\TT_{d}\setminus\{0\}$ such that $f\cdot A\subseteq M$ where $M=\oplus_{i=1}^{n}\TT_{d}a_{i}\subseteq L$. We put $\Vert\sum_{i=1}^{n}b_{i}a_{i}|M\Vert=\max\{\Vert b_{i}|\TT_{d}\Vert:1\leq i\leq n\}$. By a result from the Winter, $\Vert a|A\Vert$ and $\Vert fa|M\Vert$ are equivalent norms on $A$. Since $\Vert fa|M\Vert$ is a weakly Cartesian $K$-norm on $L$, it follows that $\Vert\cdot|A\Vert$ is the restriction to $A$ of the spectral norm on $L$. By \Cref{prop: equivalent conditions on dense embeddings}, this is equivalent to any weakly Cartesian norm on $L$ -- recall here that the maximum norm is bounded by the spectral norm. In particular, the norm restricting to $\Vert fa|M\Vert$ on $A$. Hence $\Vert\cdot|A\Vert_{s}=\Vert\cdot|A\Vert_{\max}$ as stated. The general case can be reduced to this one by induction on the number of minimal prime ideals of $A$. 
\end{proof} %end of page 1
\begin{example}
    $\Vert\cdot|A\Vert_{\max}$ may fail to be multiplicative even when $A$ is an integral domain. For example,
    let $A=\TT_{2}/(X_{1}X_{2}-\pi)$ for $\pi\in K^{\times}\cap K^{\circ\circ}$ a topoligically nilpotent unit. Then 
    \begin{align*}
        \Vert X_{1}|A\Vert_{\max}&=1 \\
        \Vert X_{2}|A\Vert_{\max}&=1 \\
        \Vert X_{1}X_{2}|A\Vert_{\max} &= |\pi|<1.
    \end{align*}
\end{example}
\begin{proposition}\label{prop: multiplicative iff reduced}
    Let $A$ be an affinoid $K$-algebra. The spectral norm on $A$ is multiplicative if and only if $A$ is reduced and $\widetilde{A}=A^{\circ}/A^{\circ\circ}$ is an integral domain. 
\end{proposition}
\begin{proof}
    $(\Rightarrow)$ Since $\Vert\cdot|A\Vert_{s}$ is power-multiplicative, the nilradical of $A$ is trivial when $\Vert\cdot|A\Vert_{s}$ is a norm. 

    $(\Leftarrow)$ Suppose $A$ is reduced and $\widetilde{A}$ is an integral domain. Let $a_{1},a_{2}\in A\setminus\{0\}$. We need to show $\Vert a_{1}a_{2}|A\Vert_{s}=\Vert a_{1}|A\Vert_{s}\cdot\Vert a_{2}|A\Vert_{s}$. Since $\Vert\cdot|A\Vert_{s}$ is power multiplicative, it suffices to prove the statement for $a_{1}^{n},a_{2}^{n}$ where $n\in\NN$. By all residual norms on an affinoid $K$-algebra being equivalent, there are $k_{i}\in K^{\times}$ such that $|k_{i}|=\Vert a_{i}|A\Vert_{s}$. Replacing $a_{i}$ by $a_{i}/k_{i}$ we have $\Vert a_{i}|A\Vert_{s}=1$ and the inequality $\Vert a_{1}a_{2}|A\Vert_{s}<1$ would imply that the images of $a_{i}$ in $\widetilde{A}$ are nonzero nilpotents. 
\end{proof}
\begin{remark}
    Let $A$ be an affinoid $K$-algebra. We can use these results to construct continuous valuations of $A$ hence showing that the dimension of $\Spa(A,A^{\circ})$ is equal to $\dim(A)$. For a domain $A$, $\TT_{d}\subseteq A$ for some $d$ by Noether normalization such that $A/\TT_{d}$ is finite. Then $\dim(A)=d$ by going up. 
\end{remark} %end of page 2
Let $E$ and $F$ be the fields of fractions of $\TT_{d}$ and $A$, respectively. Equip $E$ with the absolute value defined by the completely multiplicative $\TT_{d}$-norm. $|\cdot|_{E}$ admits an extension to $|\cdot|_{F}$. 
\begin{example}
    Let $Y=V(X_{1}X_{2}-\pi)\subseteq\Sp(\TT_{2})$. Define 
    $$|f|^{(i)}=\max\{|f(x)|:|X_{i}|=1\}$$ 
    which is an absolute value on $A=\TT_{2}/(X_{1}X_{2}-\pi)$. $F^{\circ}/F^{\circ\circ}$ is finite over $E^{\circ}/E^{\circ\circ}$ so the transcendence degree of the residue field over the ground field is $d$ and thus defines a rank $d$ valuation $\nu$ such that $\nu(\widetilde{A})\leq 1$ recalling here that $\widetilde{A}=A^{\circ}/A^{\circ\circ}$. The construction above can be used to augument $|\cdot|_{F}$ to a rank $d+1$ valuation on $F$, for example, taking 
    $$\mu(f)=\begin{cases}
        0 & f=0 \\ (|f|_{F},\nu(f\pmod{F^{\circ\circ}})) & \text{otherwise.}
    \end{cases}$$
    Its restriction to $A$ gives a rank $d+1$ valuation, that is, a point of $\Spa(A,A^{\circ})$. 
\end{example}
We recall the following propositions. 
\begin{proposition}[\Cref{prop: integral over for fg modules}]
    Let $A\to B$ be a morphism of affinoid $K$-algebras such that $B$ is finitely generated as an $A$-module. Then $\widetilde{B}$ is finitely generated over $\widetilde{A}$. 
\end{proposition}
\begin{corollary}
    If $A$ is an affinoid $K$-algebra then $\widetilde{A}$ is of finite type over $\Kfrak=K^{\circ}/K^{\circ\circ}$.
\end{corollary} %end of page 3
Recall that 
\begin{equation}\label{eqn: structure sheaf}
    \Fcal_{\Omega}(B)=\left\{f\in\Hom_{\Aff_{K}}(A,B):\Sp(f)(\Sp(B))\subseteq\Omega\subseteq\Sp(A)\right\}.
\end{equation}
This functor was used to define sections over a rational open subset, but can be applied to arbitrary open subsets of the Tate spectrum $\Sp(A)$. 
\begin{definition}[Affinoid Open Subset]\label{def: affinoid open subset}
    An open subset $\Omega\subseteq\Sp(A)$ is affinoid if the functor (\ref{eqn: structure sheaf}) is representable.
\end{definition}
\begin{definition}[Admissable Open Subset]\label{def: admissable open subset}
    An open subset of $\Omega\subseteq\Sp(A)$ is admissable if and only if for every morphism $A\to B$ in $\Aff_{K}$ the map $\Sp(B)\to\Sp(A)$ is continuous over $\Omega$. 
\end{definition}
\begin{remark}
    The continuity condition is that 
    $$\{V\in\Opens_{f^{-1}(\Omega)}:\exists U\subseteq \Sp(A)\text{ s.t. }f(V)\subseteq U\}$$ 
    is a covering sieve of $f^{-1}(\Omega)$. 
\end{remark}
We can now state without proof the Gerritzen-Grauert theorem. 
\begin{theorem}[Gerritzen-Grauert]\label{thm: GG thm}
    Let $\Omega\subseteq\Sp(A)$ be affinoid then $\Omega$ can be written as a finite union of rational open subsets of $\Sp(A)$. In particular, any affinoid open subset is a $G_{+}$-quasicompact open subset of $\Sp(A)$.
\end{theorem}
\begin{proof}
    Omitted. 
\end{proof}
We consider the simple example of a finite union of rational open subsets. 
\begin{proposition}\label{prop: finite union of rationals is admissable}
    Let $A$ be an affinoid $K$-algebra. Any finite union of rational open subsets is admissable. 
\end{proposition}
\begin{proof}
    Let $\Omega=\bigcup_{i=1}^{n}\Omega_{i}$ be such that each $\Omega_{i}\in\mathsf{Rat}_{\Sp(A)}$ and $\Scal$ a covering sieve for $\Omega$. There are finitely many $\Omega_{ij}\in\Scal\cap\Bcal_{\Omega_{i}}$ such that $\Omega_{i}=\bigcup_{j=1}^{m_{i}}\Omega_{ij}$. If $\varphi:A\to B$ inducing $f:\Sp(B)\to\Sp(A)$ and $\Omega_{ij}=R_{\Sp(A)}(a_{ij,0}|a_{ij,1},\dots,a_{ij,n})$ we have $f^{-1}(\Omega_{ij})=R_{\Sp(B)}(\varphi(a_{ij,0})|\varphi(a_{ij,1}),\dots,\varphi(a_{ij,n}))\in\mathsf{Rat}_{\Sp(B)}\cap f^{-1}\Scal$. Since $f^{-1}(\Omega)=\bigcup_{i=1}^{n}\bigcup_{j=1}^{m_{i}}\Omega_{ij}$, this shows $f^{-1}\Scal$ is a covering sieve of $f^{-1}(\Omega)$. 
\end{proof} %end of page 4
