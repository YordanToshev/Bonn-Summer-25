\section{Lecture 8 -- 17th June 2025}\label{sec: lecture 8}
We start with a 
\begin{remark}
The degree of $\tilde{A}$ over $\underline{K}$ is less-equal the Krull-dimension
of $A$.
\end{remark}
We want to pursue the problem of how to write down a rank $d+1$ valuations
on $A$, where $d=\dim(A)$. Recall from end of last term the rank 2 examples
for $\mathbb{T}_1$ reduced to inspecting $\Vert\cdot|\mathbb{T}_1\Vert$ 
or the maximum norm. In our case of interest the maximum norm will
coincide with the spectral norm, so we want to understand how to
deal with the maximum norm.
\begin{proposition}
When $A$ is reduced, then $\Vert\cdot|A\Vert_{max}$ is equivalent to the
residual norm.
\end{proposition}
\begin{proof}
Last term.
\end{proof}
As $\Vert\cdot|\mathbb{T}_1\Vert$ is multiplicative (i.e. 
$\Vert fg|\mathbb{T}_1\Vert=\Vert f|\mathbb{T}_1\Vert\Vert g|\mathbb{T}_1\Vert$),
it defines an absolute value on the field of fractions $\mathbb{K}_1$.
But $\mathbb{K}_1$ is not complete w.r.t. the unique extension
$\Vert\cdot|\mathbb{K}_1\Vert$.

Herein, we let $K$ be a field with non-Archimedean absolute value (not necessarily complete). Let $f,g\colon V\to W$ be linear maps between normed
$K$-vector spaces. We say $f>>g$ if there is $C>0$ such that $|g|\leq C\cdot|f|$. 
\begin{proposition}\label{prop: equivalent conditions on vector spaces}
    Let $V$ be a finite dimensional $K$-vector space and $\Vert\cdot\Vert_{V}$ satisfying $\Vert \lambda v\Vert_{V}=|\lambda|\cdot\Vert v\Vert_{V}, \Vert v+w\Vert_{V}\leq\max\{\Vert v\Vert,\Vert w\Vert\}$. The following conditions are equivalent: 
    \begin{enumerate}[label=(\roman*)]
        \item If $(b_{i})_{i=1}^{n}$ is any basis then $\Vert\sum_{i=1}^{n}\lambda_{i}b_{i}\Vert_{V}>>\max_{1\leq i\leq n}|\lambda_{i}|$ (viewed as linear functions in the $\lambda_i$). 
        \item For every $v\in V\setminus\{0\}$ there is a bounded linear functional $\ell:V\to K$ such that $\ell(v)\neq0$. 
    \end{enumerate}
\end{proposition}
\begin{proof}
    (a)$\Rightarrow$(b) Suppose there is some basis where (i) holds. Let $\ell_{i}\in V^{\vee}$ such that $\ell_{i}\left(\sum_{j=1}^{n}\lambda_{i}b_{i}\right)=\lambda_{i}$. These are bounded. 

    (b)$\Rightarrow$(a) We proceed by induction on the dimension of $V$. The cases where $\dim(V)\leq 1$ are clear. Let $n>1$ and suppose the assertion is shown for $\dim(V')<n$ and $v\in V\setminus\{0\}$ such that $V=V'\oplus Kv$. Let $\ell\in V^{\vee}$ be such that $\ell(v)\neq0$ with $\Vert (v',\lambda v)\Vert_{V}=\max\{|\lambda|,\Vert v'\Vert_{V}\}$ and the induction assumption applies. 
\end{proof}
\begin{remark}
    This does not hold in general when $K$ is not complete. In this case, a $K$-linear functional is bounded if and only if it is continuous. 
\end{remark}
We now consider the general case without the condition on multiplicativity of the norm. 
\begin{proposition}\label{prop: finite subspace is closed}
    Let $V$ be a normed $K$-vector space. The following are equivalent:
    \begin{enumerate}[label=(\alph*)]
        \item Every finite-dimensional subspace of $V$ satisfies the equivalent conditions of \Cref{prop: equivalent conditions on vector spaces}. 
        \item Every finite dimensional subspace of $V$ is closed in $V$. 
    \end{enumerate}
\end{proposition}
\begin{proof}
   (a)$\Rightarrow$(b) Suppose $W\subseteq V$ is a finite dimensional subspace and $v\in\overline{W}\setminus W$. Let $\widetilde{W}=W\oplus Kv$ equipped with a norm $\Vert \cdot\Vert_{V}|_{\widetilde{W}}$ the restriction of the one on $V$. For $v$ in the closure of $W$ in $\widetilde{W}$ the norms $\Vert\cdot\Vert_{V}$ and $\sum_{i=1}^{n}\lambda_{i}b_{i}\mapsto \lambda_{i}$ for a basis $(b_{i})_{i=1}^{n}$ fails to be eqiuvalent to 

   (b)$\Rightarrow$(a) Let $W\subseteq V$ be finite dimensional, $w\in W\setminus\{0\}$, and $W'$ a subspace of $W$ such that $W=W'\oplus K w$ algebraically. If $\ell:W\to K$ is a linear functional, set $\ell(\lambda w-w')=\lambda$ when $w\in W'$ and all $\lambda\in K$. This produces a linear functional nonvanishing at $w'$. 
\end{proof}
This leads to the notion of weakly Cartesian norms. 
\begin{definition}[Weakly Cartesian Norm]\label{def: weakly Cartesian norm}
    Let $V$ be a normed $K$-vector space. $V$ is weakly Cartesian if either (and thus both) of the conditions of \Cref{prop: finite subspace is closed} hold. 
\end{definition}
\begin{remark}
    By \Cref{prop: equivalent conditions on vector spaces}, two weakly Cartesian norms on any finite-dimensional vector space are equivalent. 
\end{remark}
\begin{lemma}\label{lem: weakly cartesian as K-norm}
    Let $|\cdot|$ be an ultrametric absolute value on $L$ and $K$ a subfield such that $[L:K]<\infty$ and $L$ is weakly Cartesian as a $K$-vector space. Then every weakly Cartesian $L$-norm on an $L$-vector space $V$ is weakly Cartesian as a $K$-norm. 
\end{lemma}
This shows that there are many bounded $K$-linear functionals on $L$. 
\begin{proposition}\label{prop: spectral norm}
    Let $|\cdot|$ be an ultrametric absolute value on $K$ and $L/K$ a finite field extension. Then: 
    \begin{enumerate}[label=(\roman*)]
        \item $\Vert\ell|L\Vert_{s}=\lim_{n\to\infty}\sqrt[n]{\Vert\ell\Vert^{n}}$ does not depend on the choice of a weakly Cartesian $K$-norm $\Vert\cdot\Vert$ on $L$ and is a power multiplicative ring norm and a $K$-vector space norm on $L$. 
        \item Every power multiplicative norm on $L$ is bounded by the spectral norm $\Vert\cdot\Vert_{s}$. 
        \item If the minimal polynomial of $s$ is of the form $T^{d}+\sum_{i=0}^{d-1}p_{i}T^{i}$ then $\Vert\ell\Vert=\max_{0\leq i<d}\sqrt[d-i]{|p_{i}|}$. 
    \end{enumerate}
\end{proposition}
\begin{proof}[Proof of (i)]
    If $\Vert\cdot\Vert$ is given by a basis $(b_{i})_{i=1}^{n}$ as in \Cref{prop: equivalent conditions on vector spaces} and $b_{i}b_{j}=m_{ij}^{k}b_{k}$, then $\Vert xy\Vert\leq C\Vert x\Vert\cdot\Vert y\Vert$ with $C=\max\{m_{ij}^{k}\}$. Such a $C$ exists by being weakly Cartesian. We have that $\Vert x^{n}\Vert\leq C^{k}\Vert x\Vert^{n}$ and applying this iteratively by induction we have $k$ depends on $n$ sublinearly so $n\leq 2^{k}$ and we have $\lim_{n\to\infty}\sqrt[n]{\Vert x\Vert^{n}}\leq\sqrt[n]{\Vert x^{m}\Vert}$ when $m>0$ showing convergence. Checking that the axioms of the norm hold shows that $\Vert x\Vert_{s}\neq0$ when $x\neq0$. 
\end{proof}
\begin{proof}[Proof of (ii)]
    If $N$ is any norm and $\Vert\cdot\Vert$ is a weakly Cartesian norm, then $N(\ell)\leq\Vert \ell\Vert$ -- every norm is bounded by a weakly Cartesian one, but not conversely. 
\end{proof}
\begin{proof}[Proof of (iii)]
    Let $M=\max_{0\leq i<d}\sqrt[d-i]{|p_{i}|}$ then $\Vert x\Vert_{s}^{n}=\Vert x^{n}\Vert\leq M^{n+1-d}$ where $n\geq d$ by induction on $n$. Hence $\Vert\cdot\Vert_{s}\leq M$. For the reverse inequality, the arguments of (i) give that without loss of generality we can take $L/K$ to be normal. If $(x_{i})_{i=1}^{m}$ are the images of $x$ under the elements of $\Aut(L/K)$ then the minimal polynomial of $x$ over $K$ is of the form $\left(\prod_{i=1}^{a}(T-x_{i})\right)^{q}$ for some positive integer $q$ as $\Vert x_{i}\Vert_{s}=\Vert x\Vert_{s}$ by (i) so $M\leq\Vert x\Vert_{s}$ giving the equality. 
\end{proof}
We will consider $\Vert\cdot\Vert_{s}$ for algebraic extensions of $L$ over $K$. 
\begin{proposition}\label{prop: spectral norm for field extensions}
    Let $L/K$ be a finite field extension and $|\cdot|$ an ultrametric absolute value on $K$. 
    \begin{enumerate}[label=(\roman*)]
        \item There is a bijection between extensions of $|\cdot|$ to some absolute value of $L$ and the isomorphism classes of embeddings $L\to\widehat{L}$ with dense image and $\widehat{L}$ finite over $\widehat{K}$ by $|\cdot|_{L}\mapsto\widehat{L}$ the completion of $L$ with respect to that absolute value and conversely $L\hookrightarrow\widehat{L}$ to the unique norm $|\cdot|_{L}$ where $|\ell|_{L}=|\ell|_{\widehat{L}}$. 
        \item For a list of representatives of isomorphism classes in (i) and $d_{i}=[\widehat{L}_{i}:\widehat{K}]$ then $\sum_{i=1}^{n}d_{i}<[L:K]$. In particular, $n$ is finite and the spectral norm admits an explicit description $\Vert x\Vert_{s}=\max_{1\leq i\leq n}|x|_{\widehat{L}_{i}}$. 
    \end{enumerate}
\end{proposition}
\begin{proof}[Proof of (i)]
    This is immediate from the construction. 
\end{proof}
\begin{proof}[Proof of (ii)]\todo{Check}
    Let $\LL$ be the completion of $L$ with respect to any weakly Cartesian norm. This is a vector space over $\widehat{K}$ of dimension equal to the degree of the field extension $[L:K]$ and $\LL\twoheadrightarrow\bigoplus_{i=1}^{n}\widehat{L}_{i}$ is reduced. That $\Vert x\Vert_{s}\geq\Vert x\Vert_{\widehat{L}_{i}}$ follows from \Cref{prop: spectral norm} (ii). For the opposite inequality, we can take $L/K$ to be a normal field extension, wherein $\Aut(L/K)$ permutes $L$ transitively. For $n>0$ let $\widehat{L}$ be a finite extension of $\widehat{K}$ in which $L$ can be embedded. Then this follows from \Cref{prop: spectral norm} (iii). 
\end{proof}
