\section{Lecture 4 -- 15th May 2025}\label{sec: lecture 4}
We consider the van der Put points of a Tate spectrum in terms of the adic spectrum. 

The construction of the adic spectrum begins with Huber pairs. 
\begin{definition}[Huber Pair]\label{def: Huber pair}
    A Huber pair is a pair $(A,A^{+})$ where:
    \begin{itemize}
        \item $A$ is a nat ring with an open bounded subring hose topology is $I$-adic for some finitely generated ideal $I$. 
        \item $A^{+}\subseteq A^{\circ}$ is an open integrally closed subring. 
    \end{itemize}
\end{definition}
\begin{definition}[Morphism of Huber Pairs]\label{def: morphism of Huber pairs}
    A morphism of Huber pairs $\varphi:(A,A^{+})\to(B,B^{+})$ is a continuous homomorphism of rings $\varphi:A\to B$ such that $\varphi(A^{+})\subseteq B^{+}$. 
\end{definition}
\begin{remark}
    The subring of powerbounded elements $A^{\circ}$ is always an integrally closed subring of $A$. In particular, if $A$ is Huber, $(A,A^{\circ})$ is a Huber pair. If further $A$ is Tate, then for any morphism of Huber pairs $\varphi:(A,A^{\circ})\to (B,B^{\circ})$ the condition that $\varphi(A^{\circ})\subseteq B^{\circ}$ is automatic and $B$ is also Tate. 
\end{remark}
The adic spectrum is defined as a certain subspace of the space of valuations. 
\begin{definition}[Valuation]\label{def: valuation}
    A valuation on a ring $A$ is a map $\nu:A\to\Gamma\cup\{0\}$, where $\Gamma$ is written multiplicatively, such that:
    \begin{enumerate}[label=(\roman*)]
        \item $0<\gamma$ for all $\gamma\in\Gamma$. 
        \item $\nu(ab)=\nu(a)\cdot\nu(b)$. 
        \item $\nu(a+b)\leq\max\{\nu(a),\nu(b)\}$. 
        \item $\nu(1)=1$. 
    \end{enumerate}
\end{definition}
\begin{definition}[Support of Valuation]\label{def: support of valuation}
    Let $\nu$ be a valuation on a ring $A$. The support $\supp(\nu)$ of $\nu$ is the prime ideal $\{a\in A:\nu(a)=0\}\subseteq A$. 
\end{definition}
\begin{remark}\label{rmk: equivalence relation on valuations}
    We assume that $\Gamma$ is generated by the image of $R\setminus\supp(\nu)$. In this case, $\nu\simeq\widetilde{\nu}$ if and only if there is a unique isomorphism of groups $\tau:\Gamma\to\widetilde{\Gamma}$ of groups $\widetilde{\nu}=\tau\circ\nu$. 
\end{remark}
We consider some additional constructions related to valuations. 
\begin{definition}[Convex Subset]\label{def: convex subset}
    A subset $X$ of $\Gamma$ is convex if and only if for all $\gamma,\gamma'\in X$, $[\gamma,\gamma']_{\Gamma}\subseteq X$. 
\end{definition}
\begin{definition}[Rank of Valuation]\label{def: rank of valuation}
    Let $\nu$ be a valuation on a ring $A$. The rank of $\nu$ is the number of convex subgroups. 
\end{definition}
\begin{remark}
    The set of convex subgroups is linearly ordered by inclusion. 
\end{remark}
We can define continuous valuations by topologizing $\Gamma\cup\{0\}$ with the topology that all elements of $\Gamma$ are open points and the collection of half-open intervals $\{[0,\gamma)_{\gamma}:\gamma\in\Gamma\}$ form a neighborhood basis of 0. 
\begin{definition}[Continuous Valuation]\label{def: continuous valuation}
    Let $\nu$ be a valuation on a ring $A$. $\nu$ is a continuous valuation if it is continuous as a map where $\Gamma\cup\{0\}$ is equipped with the topology where all elements of $\Gamma$ are open points and the collection of half-open intervals $\{[0,\gamma)_{\Gamma}:\gamma\in\Gamma\}$ form a neighborhood basis of 0.
\end{definition}
\begin{remark}
    The following conditions are equivalent to a valuation being continuous: 
    \begin{itemize}
        \item If $\nu(a)\neq0$ then $\{a'\in A:\nu(a')<\nu(a)\}$ is open. 
        \item The set $\{a\in A:\nu(a)<\gamma\}$ is open for $\gamma\in\Gamma$. 
    \end{itemize}
\end{remark}
\begin{definition}[Space of Continuous Valuations]\label{def: space of continuous valuations}
    Let $A$ be a nat ring and let $\Cont(A)$ be the set of continuous valuations equipped with the topology that the rational open subsets 
    $$R_{\Cont(A)}(a_{0}|a_{1},\dots,a_{n})=\left\{\nu/\sim,\nu\text{ cts.}:\nu(a_{0})\neq0,\nu(a_{i})\leq\nu(a_{0})\text{ }\forall 1\leq i\leq n\right\}$$
    such that $(a_{0},\dots,a_{n})$ is an open ideal in $A$. 
\end{definition}
\begin{remark}
    In particular if $A$ is Tate then the only open ideal is the ring itself, and $(a_{0},\dots,a_{n})=A$. 
\end{remark}
\begin{remark}
    For general topological rings, these rational open subsets may fail to be closed under finite intersection, but this always holds for Huber rings. 
\end{remark}
We recall the following result. 
\begin{proposition}\label{prop: cont is spectral}
    Let $A$ be a nat ring and let $\Cont(A)$ be the space of continuous valuations of $A$. $\Cont(A)$ is a spectral space with quasicompact basis given by the rational open subsets. 
\end{proposition}
This allows us to construct the adic spectrum. 
\begin{corollary}\label{corr: adic spectrum is spectral}
    Let $A^{+}$ be an integrally closed subring of a nat ring $A$. The subspace
    $$\bigcap_{a\in A^{+}}R_{\Cont(A)}(1|a)=\{\nu/\sim,\nu\text{ cts.}:\nu(a)\leq 1\text{ }\forall a\in A^{+}\}$$
    is spectral.
\end{corollary}
\begin{proof}
    This is a proconstructible subset of a spectral space, hence spectral. 
\end{proof}
We can finally define the adic spectrum. 
\begin{definition}[Affinoid Adic Space]\label{def: affinoid adic space}
    Let $(A,A^{+})$ be a Huber pair. The adic spectrum $\Spa(A,A^{+})$ of $(A,A^{+})$ is the subspace 
    $$\bigcap_{a\in A^{+}}R_{\Cont(A)}(1|a)=\{\nu/\sim,\nu\text{ cts.}:\nu(a)\leq 1\text{ }\forall a\in A^{+}\}.$$
\end{definition}
\begin{remark}
    When $A$ is a Tate ring, 
    $$R_{\Cont(A)}(f_{0}|f_{1},\dots,f_{n})=\{\nu/\sim,\nu\text{ cts.}:\nu(f_{i})\leq\nu(f_{0})\text{ }\forall 1\leq i\leq n\}.$$
\end{remark}
Henceforth we take $A$ to be an affinoid $K$-algebra. 
\begin{example}
    If $\kappa\in K$ and $|\kappa|=1$ with $K^{\circ}\subseteq A^{\circ}$ and $1/\kappa\in K^{\circ}\subseteq A^{\circ}$ so $\nu(\kappa)\leq 1,\nu(1/\kappa)\leq 1$ showing $\nu(\kappa)=1$ and $K^{\times}$ is a subgroup of $\Gamma$. 
\end{example}
Up to replacing $\nu$ by an equivalent valuation as in \Cref{rmk: equivalence relation on valuations}, we can assume that $|K^{\times}|\subseteq\Gamma$ which is typically not convex. If $t\in\RR$ such that $t^{n}\in|K^{\times}|$ for some $n\in\NN$, say $\gamma\leq t$ (resp. $\gamma<t$) if and only if $\gamma^{n}\leq t^{n}$ (resp. $\gamma^{n}<t^{n}$). If $t\in\RR_{>0}$ and there is no positive integer $n$ with $t^{n}\in |K^{\times}|$ exists, say $\gamma\leq t$ if and only if $\gamma<t$ if and only if there exists $n\in\NN$ and $\kappa\in K^{\times}$ such that $\gamma^{r}\leq|\kappa|^{r}<t$. For general real numbers $t,\gamma$, the inequality $t<\Gamma$ is dealt with in the same way. 

In what follows, we will use the following fact. 
\begin{lemma}
    Let $A$ be an affinoid $K$-algebra. If $\Vert-|A\Vert$ is a residual norm, then a valuation $\nu$ on $A$ such that $\nu((K^{\circ})^{\times})\subseteq\{1\}$. $\nu$ is continuous if and only if there exists $c\in\RR$ such that $\nu(a)\leq c\Vert a|A\Vert$ and $|K^{\times}|$ is cofinal in $\Gamma$ -- for all $\gamma\in\Gamma$, there is $\varepsilon\in K^{\times}$ such that $|\varepsilon|\leq\gamma$. 
\end{lemma}
We will soon show that the van der Put points of the Tate $\Sp(A)$ is homeomorphic as a $G_{+}$-space to $\Spa(A,A^{\circ})$ by constructing a specific bijection, and show that $\Spa(A,A^{\circ})$ has Krull dimension is that of $A$. 
\begin{example}
    Let $A=\TT_{1}$ and $A^{+}=K^{\circ}+A^{\circ\circ}$. $(A,A^{+})$ is a Huber pair and $\Spa(A,A^{\circ})\setminus\Spa(A,A^{+})$ has exactly one element $\nu$ where $\nu(f)=(\Vert f|\TT_{1}\Vert -\kappa_{f})$ where for $f=\sum_{j\geq0}f_{j}T^{j}\in\TT_{1}\setminus\{0\}$, $\kappa_{f}=\max\{j:|f_{j}|=\Vert f|\TT_{1}\Vert\}$. 
\end{example}
\begin{remark}
    The pair $(A,A^{+})$ is not necessarily topologically of finite type over $(K,K^{\circ})$. 
\end{remark}
The proofs of the abovementioned result will require a careful treatment of powerbounded elements and the proof of the result on Krull dimension is also fairly subtle. 